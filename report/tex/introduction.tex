\section{Introduction}

Dans ce projet, nous devions mettre en place une simulation de livraison par drones en mettant en avant la problématique
de la synchronisation entre les différents acteurs. Cela dans le but de mettre en pratique concrètement des outils tels que
les sémaphores, les processus, les threads, les signaux, …

Nous avons choisi de répondre au problème en utilisant :
\begin{itemize}
    \item des processus et non des threads
    \item des signaux
    \item des tubes (pipes)
    \item des sémaphores
    \item de la mémoire partagée
    \item une file de messages
\end{itemize}

L'utilisation respective de ces différents éléments est détaillée par la suite du rapport.

Notre simulation prend en charge un certain nombre de fonctionalités :
\begin{itemize}
    \item configuration de la simulation via fichier csv.
    \item arguments à passer en ligne de commandes pour l'exécution de la simulation (cf \emph{README.md}).
    \item La simulation tourne en tenant compte
        \begin{itemize}
            \item des drones : autonomie de fonctionnement, distance à couvrir,
                poids et du volume pouvant être stocké, disponibilité.
            \item des colis : priorité du colis, poids et volume du colis.
            \item des capacités du vaisseau mère : emplacements de recharge disponibles,
                vitesse de chargement des colis sur les drones, …
            \item de la disponibilité des clients.
            \item des couloirs aériens : tous les drones ne peuvent pas partir en même temps dans la même direction.
            \item d'éventuels aléas pouvant survenir : ici représentés par des « chasseurs » qui tirent sur les drones.
        \end{itemize}
\end{itemize}

Lorsque la simulation est terminée, un rapport de simulation est affiché.

Nous allons d'abord étudier le problème à l'aide de la modélisation d'un diagramme de Pétri.

