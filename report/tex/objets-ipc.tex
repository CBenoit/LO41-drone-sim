\section{Synchronisation et communication}

Tout d'abord, nous avons fait le choix d'utiliser les processus et non les threads.
Le vaisseau mère est le processus père de tous les autres processus : processus drone, processus client et processus chasseur.
De ce fait, tous les éléments sont réellement indépendants les uns des autres et il est nécessaire de communiquer via
des objets IPC : pas de partage de l'espace d'adressage comme pour les threads (qui n'auraient alors nécessité que
la mise en place de verroux pour la synchronisation). Cela nous a semblé plus proche
du comportement final du système de livraison. Il est possible d'envoyer manuellement le signal SIGKILL à un processus
drone et la simulation s'adaptera, de la même façon que dans la réalité un drone peut avoir dysfonctionnement.
Des événements tels que les chasseurs (processus tueurs de drones) sont également présents pour simuler cela.
La gestion des signaux avec les processus est bien plus commode.

Pour la synchronisation de la simulation, nous avons mis en place un système d'horloge et de pas de simulation. Un pas
de simulation correspond à un tic d'horloge. Lors un pas de simulation, les différents processus effectuent des tâches
bien précises et attendent le pas suivant. Ainsi, on est sûrs que tout le monde effectue à temps équivalent la même
quantité d'action. De plus, il est possible de ralentir ou d'accelerer la simulation selon la vitesse de l'horloge.

Nous utilisons la plupart des objets IPC : \emph{signaux, tubes, sémaphores, mémoire partagée et file de messages.}

Les \emph{signaux} sont utilisés dans diverses portions du programme :
\begin{itemize}
    \item gestion de l'interruption clavier.
    \item meurtre de drone (par les chasseurs) avec SIGKILL.
    \item écoute des processus enfants qui s'arrêtent par le vaisseau mère (cf: chapitre \ref{chap:mothership}).
    \item indication à un processus précis de se remettre à travailler (exécution d'un tic d'horloge) avec un signal utilisateur
        envoyé par le vaisseau mère (cf: chapitre \ref{chap:mothership}).
\end{itemize}

Nous utilisons une seule \emph{sémaphore} pour permettre au vaisseau mère d'attendre les processus qui travaillent. Lorsqu'un
processus a terminé son tic d'horloge, il incrémente la valeur de la sémaphore.

Les \emph{tubes} sont utilisés pour la communication drones/clients.
Les clients ouvrent tous une paire de tubes (un pour la lecture, l'autre pour l'écriture).
Lorsqu'un drone veut communiquer avec un client, il écrit simplement dans le bon tube et attend la réponse dans l'autre tube.

Le \emph{segment de mémoire partagée} est utilisé pour compter et stocker une liste de drones actuellement dans les airs.
Le vaisseau mère tient cette liste à jour et les chasseurs piochent dans la liste pour tuer des drones aléatoirement.

La \emph{file de messages} est utilisée pour la communication vaisseau mère/drones. Le type des messages correspond au pid des
destinataire. Ainsi, chacun peut récupérer les messages qui lui sont destinés et travailler selon.

