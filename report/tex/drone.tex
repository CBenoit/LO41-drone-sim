\section{Fonctionement des drones}

Chaque drone est un processus à part entière, de la même manière que le vaisseau mère, les clients, et les chasseurs.
Tout les drônes attendent l'arrivée du tic du vaisseau mère, et agissent alors parallèlement, selon leur état courant.
Les drone sont en effet des \emph{Automates fini}, mis en place à l'aide de pointeurs vers des fonctions.

Les différents états sont les suivants:
\begin{itemize}
    \item en vol
    \item livraison du colis
    \item rechargement en carburant
    \item chargement du colis
    \item attente d'autorisation de départ.
\end{itemize}

\subsection{État `En vol'}
    Lorsqu'un drone est en vol, il consomme une unité de carburant par tic, et avance de sa N unités de distance vers sa cible,
    le client ou le vaisseau mère (avec N la vitesse du drone).
    Si sa réserve de caburant arrive à \(0\), le drone se crashe alors via un \codequote{exit(CRASHED)},
    et le vaisseau mère peut alors récupérer l'information avec un \codequote{wait}. \\
    Une fois que le vol est terminé, le drone signale au vaisseau mère son arrivée via une file de messages,
    et passe dans l'état `livraison du colis' si le drone est parvenu au client qu'il doit livrer,
    ou `rechargement en carburant' s'il vient d'arriver au vaisseau mère.

\subsection{État `Livraison du colis'}
    Dans l'état `livraison du colis', le drone se pose chez son client et l'informe de son arrive.
    Après cela, il reste en attente jusqu'à ce que son client cible ait finit de récupérer son colis.
    Une fois son client livré, il passe à l'état `Attente d'autorisation de départ', afin de ne pas
    percuter un autre drone qui utiliserai le même couloir aérien.

\subsection{État `Attente d'autorisation de départ'}
    Avant de décoller du vaisseau mère ou de revenir de chez le client, le drone demande au vaisseau mère
    l'autorisation de décoller, pour empêcher deux accidents possible:
    \begin{itemize}
        \item collision entre deux appareils utilisant le même couloir aérien
        \item un client n'a pas fini de récupérer son colis (on ne sait pas combien de temps il va encore prendre).
    \end{itemize}
    Un fois qu'il a obtenu l'autorisation de partir, le drone s'en va vers sa destination, et passe dans l'état `en vol'.

\subsection{État `Rechargement en carburant'}
    Pour se recharger en carburant, le drone demande tout d'abord au vaisseau mère un câble de recharge.
    S'il n'y en a pas de disponible, le drone redemande alors câble au tic suivant.
    Une fois de câble obtenu, le drone attend d'être completement chargé, libère le câble,
    et passe dans l'état `chargement du colis'.

\subsection{État `Chargement du colis'}
    Dans cet état, le drône demande au vaisseau mère de lui fournir un colis.
    Il y a alors trois possibilités:
    \begin{itemize}
        \item aucun bras de chargement n'est dispnible, auquel cas le drone attend le tic suivant
        \item aucun colis adapté au drone n'est disponible, auquel cas le drone est éteind.
        \item un colis est disponible, et est alors chargé sur le drone.
    \end{itemize}
    Si le drone a acqui un colis, il passe alors dans l'état `Attente d'autorisation de départ',
    avant de s'élancer vers son client cible.

