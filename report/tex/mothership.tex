\chapter{Fonctionement du vaisseau mère}\label{chap:mothership}

Le processus du vaisseau mère est le parent de tous les autres processus de type drone, client ou chasseur.
Il connaît le pid de tout le monde et est en mesure de surveiller l'état de tout le monde. Il écoute le signal SIGCHLD pour
être notifié si un processus meurt et de quelle façon : selon ce qui se produit, il s'assure que la simulation peut
continuer sans problème en mettant à jour certaines données ou bien stoppe la simulation et nettoie la mémoire.
Il s'occupe aussi de gérer l'interruption (ctrl+C) : nettoyage de la mémoire et envoi d'un signal d'extinction aux enfants.

C'est aussi lui qui gère la synchronisation et le pas de simulation :
nous utilisons une sémaphore POSIX pour faire dormir le vaisseau mère et attendre qu'un certain nombre de processus aient
terminé leur pas de simulation. Puis, le vaisseau mère lit les messages qui lui sont destinés et effectue les
tâches associées. Une fois la lecture des messages terminée, le vaisseau mère envoie un signal
aux processus qui peuvent se remettre à travailler.

Le vaisseau mère fait travailler les processus dans un ordre bien précis : c'est d'abord les processus drones qui
se remettent à travailler, puis les processus clients et enfin les processus chasseurs.
L'ordre a une certaine importance dans la mesure où :
\begin{itemize}
    \item les chasseurs ne peuvent tirer sur un drone que si celui-ci est dans les airs.
    \item si un drone veut communiquer avec le client, la lecture étant non bloquante (pour les besoins de l'horloge de simulation),
        le plus efficace est de faire en sorte d'envoyer le message en premier. Ainsi, l'envoie et la lecture du message a lieu
        au même tic d'horloge. Ceci a du sens dans la mesure où ce n'est jamais le client qui initie la communication avec le
        drone, mais toujours le drone qui initie la communication avec le client.
\end{itemize}

Le vaisseau mère et les drones communiquent également via une file de messages System V qui,
contrairement à la file de messages POSIX, permet de donner un type aux messages. Nous utilisons
le type pour indiquer le pid du destinataire du message.

Au niveau de la simulation en elle même, le vaisseau mère s'occupe d'un certain nombre de tâches importantes :
\begin{itemize}
    \item autoriser le \emph{décollage et le départ des drones}. Pour pouvoir partir, un drone doit nécessairement envoyer
        une requête au vaisseau mère qui pourra ou non l'autoriser à partir selon la disponibilité des couloirs aériens.
    \item fournir des emplacements de \emph{recharche de batterie} aux drones qui le demandent.
    \item fournir un \emph{colis approprié} aux drones. Lorsqu'un drone demande un colis, le vaisseau mère sélectionne un colis
        selon plusieurs critères : le contenance et le poids maximal que le drone peut porter ainsi que son autonomie.
        De plus, sur plusieurs colis appropriés, c'est celui dont la priorité est la plus élevée qui est sélectionné.
    \item \emph{garder une trace des différents évènements} qui se produisent :
        colis livrés, à livrer et perdus, couloirs aériens utilisés, drones disponibles, en vol, perdus, …
\end{itemize}

La liste des drones en vol est dans un segment de mémoire partagée (objet POSIX) et c'est le vaisseau mère
qui s'assure de la tenir à jour. Aucun verrou n'est nécessaire sur les modifications de cette liste car le vaisseau mère
ne la modifie que lorsque tous les autres processus sont endormis en attente du signal de reprise. La synchronisation est donc
déjà gérée.

